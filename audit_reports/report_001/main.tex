%%%%%%%%%%%%%%%%%%%%%%%%%%%%%%%%%%%%%%%%%
% Wenneker Assignment
% LaTeX Template
% Version 2.0 (12/1/2019)
%
% This template originates from:
% http://www.LaTeXTemplates.com
%
% Authors:
% Vel (vel@LaTeXTemplates.com)
% Frits Wenneker
%
% License:
% CC BY-NC-SA 3.0 (http://creativecommons.org/licenses/by-nc-sa/3.0/)
% 
%%%%%%%%%%%%%%%%%%%%%%%%%%%%%%%%%%%%%%%%%

%----------------------------------------------------------------------------------------
%	PACKAGES AND OTHER DOCUMENT CONFIGURATIONS
%----------------------------------------------------------------------------------------

\documentclass[11pt]{scrartcl} % Font size

%%%%%%%%%%%%%%%%%%%%%%%%%%%%%%%%%%%%%%%%%
% Xyz.
%
% - JLaw.
%%%%%%%%%%%%%%%%%%%%%%%%%%%%%%%%%%%%%%%%%

%----------------------------------------------------------------------------------------
%	PACKAGES
%----------------------------------------------------------------------------------------

\usepackage{listings} % Code listings with syntax highlighting.

\usepackage[english]{babel} % English hyphenation.

\usepackage{graphicx} % Insert image.
\graphicspath{{Figures/}{./}} % Specify images location.

\numberwithin{equation}{section} % Number equations within sections.
\numberwithin{figure}{section} % Number figures within sections.

\setlength\parindent{0pt} % Remove indentation from paragraphs.

\usepackage{enumitem} % List customization.
\setlist{noitemsep} % No spacing between list item.

%----------------------------------------------------------------------------------------
%	DOCUMENT MARGINS
%----------------------------------------------------------------------------------------

\usepackage{geometry} % Adjust page dimensions and margins.

\geometry{
	paper=a4paper, % Paper size.
	top=2.5cm, % Top margin.
	bottom=3cm, % Bottom margin.
	left=3cm, % Left margin.
	right=3cm, % Right margin.
	headheight=0.75cm, % Header height.
	footskip=1.5cm, % Space from bottom margin to baseline of footer.
	headsep=0.75cm, % Space from top margin to baseline of header.
}

%----------------------------------------------------------------------------------------
%	FONTS
%----------------------------------------------------------------------------------------

\usepackage[utf8]{inputenc} % Input international characters.
\usepackage[T1]{fontenc} % 8-bit encoding.

\usepackage{fourier}

%----------------------------------------------------------------------------------------
%	SECTION TITLES
%----------------------------------------------------------------------------------------

\usepackage{sectsty} % Customize section commands.

\sectionfont{\vspace{6pt}\centering\normalfont\scshape} % \section{} styling.
\subsectionfont{\normalfont\bfseries} % \subsection{} styling.
\subsubsectionfont{\normalfont\itshape} % \subsubsection{} styling.
\paragraphfont{\normalfont\scshape} % \paragraph{} styling.

%----------------------------------------------------------------------------------------
%	HEADERS AND FOOTERS
%----------------------------------------------------------------------------------------

\usepackage{scrlayer-scrpage} % Customize headers and footers.

\ohead*{} % Right header.
\ihead*{} % Left header.
\chead*{} % Centre header.

\ofoot*{} % Right footer.
\ifoot*{} % Left footer.
\cfoot*{\pagemark} % Centre footer. % Include the file specifying the document structure and custom commands

%----------------------------------------------------------------------------------------
%	TITLE SECTION
%----------------------------------------------------------------------------------------

\title{	
	\normalfont\normalsize
	\textsc{University, School or Department Name}\\ % Your university, school and/or department name(s)
	\vspace{25pt} % Whitespace
	\rule{\linewidth}{0.5pt}\\ % Thin top horizontal rule
	\vspace{20pt} % Whitespace
	{\huge Assignment Title}\\ % The assignment title
	\vspace{12pt} % Whitespace
	\rule{\linewidth}{2pt}\\ % Thick bottom horizontal rule
	\vspace{12pt} % Whitespace
}

\author{\LARGE Arthur Morgan} % Your name

\date{\normalsize\today} % Today's date (\today) or a custom date

\begin{document}

\maketitle % Print the title

%----------------------------------------------------------------------------------------
%	FIGURE EXAMPLE
%----------------------------------------------------------------------------------------

\section{Image Interpretation}

\begin{figure}[h] % [h] forces the figure to be output where it is defined in the code (it suppresses floating)
	\centering
	\includegraphics[width=0.5\columnwidth]{swallow.jpg} % Example image
	\caption{European swallow.}
\end{figure}

%------------------------------------------------

\subsection{What is the airspeed velocity of an unladen swallow?}

While this question leaves out the crucial element of the geographic origin of the swallow, according to Jonathan Corum, an unladen European swallow maintains a cruising airspeed velocity of \textbf{11 metres per second}, or \textbf{24 miles an hour}. The velocity of the corresponding African swallows requires further research as kinematic data is severely lacking for these species.

%----------------------------------------------------------------------------------------
%	TEXT EXAMPLE
%----------------------------------------------------------------------------------------

\section{Understanding Text}

\subsection{How much wood would a woodchuck chuck if a woodchuck could chuck wood?}

%------------------------------------------------

\subsubsection{Suppose ``chuck" implies throwing.}

According to the Associated Press (1988), a New York Fish and Wildlife technician named Richard Thomas calculated the volume of dirt in a typical 25--30 foot (7.6--9.1 m) long woodchuck burrow and had determined that if the woodchuck had moved an equivalent volume of wood, it could move ``about \textbf{700 pounds (320 kg)} on a good day, with the wind at his back".

%------------------------------------------------

\subsubsection{Suppose ``chuck" implies vomiting.}

A woodchuck can ingest 361.92 cm\textsuperscript{3} (22.09 cu in) of wood per day. Assuming immediate expulsion on ingestion with a 5\% retainment rate, a woodchuck could chuck \textbf{343.82 cm\textsuperscript{3}} of wood per day.

%------------------------------------------------

\paragraph{Bonus: suppose there is no woodchuck.}

Fusce varius orci ac magna dapibus porttitor. In tempor leo a neque bibendum sollicitudin. Nulla pretium fermentum nisi, eget sodales magna facilisis eu. Praesent aliquet nulla ut bibendum lacinia. Donec vel mauris vulputate, commodo ligula ut, egestas orci. Suspendisse commodo odio sed hendrerit lobortis. Donec finibus eros erat, vel ornare enim mattis et.

%----------------------------------------------------------------------------------------
%	EQUATION EXAMPLES
%----------------------------------------------------------------------------------------

\section{Interpreting Equations}

\subsection{Identify the author of Equation \ref{eq:bayes} below and briefly describe it in English.}

\begin{align} 
	\label{eq:bayes}
	\begin{split}
		P(A|B) = \frac{P(B|A)P(A)}{P(B)}
	\end{split}					
\end{align}

Lorem ipsum dolor sit amet, consectetur adipiscing elit. Praesent porttitor arcu luctus, imperdiet urna iaculis, mattis eros. Pellentesque iaculis odio vel nisl ullamcorper, nec faucibus ipsum molestie. Sed dictum nisl non aliquet porttitor. Etiam vulputate arcu dignissim, finibus sem et, viverra nisl. Aenean luctus congue massa, ut laoreet metus ornare in. Nunc fermentum nisi imperdiet lectus tincidunt vestibulum at ac elit. Nulla mattis nisl eu malesuada suscipit.

%------------------------------------------------

\subsection{Try to make sense of some more equations.}

\begin{align} 
	\begin{split}
		(x+y)^3 &= (x+y)^2(x+y)\\
		&=(x^2+2xy+y^2)(x+y)\\
		&=(x^3+2x^2y+xy^2) + (x^2y+2xy^2+y^3)\\
		&=x^3+3x^2y+3xy^2+y^3
	\end{split}					
\end{align}

Lorem ipsum dolor sit amet, consectetuer adipiscing elit. 
\begin{align}
	A = 
	\begin{bmatrix}
		A_{11} & A_{21} \\
		A_{21} & A_{22}
	\end{bmatrix}
\end{align}
Aenean commodo ligula eget dolor. Aenean massa. Cum sociis natoque penatibus et magnis dis parturient montes, nascetur ridiculus mus. Donec quam felis, ultricies nec, pellentesque eu, pretium quis, sem.

%----------------------------------------------------------------------------------------
%	LIST EXAMPLES
%----------------------------------------------------------------------------------------

\section{Viewing Lists}

\subsection{Bullet Point List}

\begin{itemize}
	\item First item in a list 
		\begin{itemize}
		\item First item in a list 
			\begin{itemize}
			\item First item in a list 
			\item Second item in a list 
			\end{itemize}
		\item Second item in a list 
		\end{itemize}
	\item Second item in a list 
\end{itemize}

%------------------------------------------------

\subsection{Numbered List}

\begin{enumerate}
	\item First item in a list 
	\item Second item in a list 
	\item Third item in a list
\end{enumerate}

%----------------------------------------------------------------------------------------
%	TABLE EXAMPLE
%----------------------------------------------------------------------------------------

\section{Interpreting a Table}

\begin{table}[h] % [h] forces the table to be output where it is defined in the code (it suppresses floating)
	\centering % Centre the table
	\begin{tabular}{l l l}
		\toprule
		\textit{Per 50g} & \textbf{Pork} & \textbf{Soy} \\
		\midrule
		Energy & 760kJ & 538kJ\\
		Protein & 7.0g & 9.3g\\
		Carbohydrate & 0.0g & 4.9g\\
		Fat & 16.8g & 9.1g\\
		Sodium & 0.4g & 0.4g\\
		Fibre & 0.0g & 1.4g\\
		\bottomrule
	\end{tabular}
	\caption{Sausage nutrition.}
\end{table}

%------------------------------------------------

\subsection{The table above shows the nutritional consistencies of two sausage types. Explain their relative differences given what you know about daily adult nutritional recommendations.}

Lorem ipsum dolor sit amet, consectetur adipiscing elit. Praesent porttitor arcu luctus, imperdiet urna iaculis, mattis eros. Pellentesque iaculis odio vel nisl ullamcorper, nec faucibus ipsum molestie. Sed dictum nisl non aliquet porttitor. Etiam vulputate arcu dignissim, finibus sem et, viverra nisl. Aenean luctus congue massa, ut laoreet metus ornare in. Nunc fermentum nisi imperdiet lectus tincidunt vestibulum at ac elit. Nulla mattis nisl eu malesuada suscipit.

%----------------------------------------------------------------------------------------
%	CODE LISTING EXAMPLE
%----------------------------------------------------------------------------------------

\section{Reading a Code Listing}

\lstinputlisting[
	caption=Luftballons Perl Script., % Caption above the listing
	label=lst:luftballons, % Label for referencing this listing
	language=Perl, % Use Perl functions/syntax highlighting
	frame=single, % Frame around the code listing
	showstringspaces=false, % Don't put marks in string spaces
	numbers=left, % Line numbers on left
	numberstyle=\tiny, % Line numbers styling
	]{luftballons.pl}

%------------------------------------------------

\subsection{How many luftballons will be output by the Listing \ref{lst:luftballons} above?}

Aliquam arcu turpis, ultrices sed luctus ac, vehicula id metus. Morbi eu feugiat velit, et tempus augue. Proin ac mattis tortor. Donec tincidunt, ante rhoncus luctus semper, arcu lorem lobortis justo, nec convallis ante quam quis lectus. Aenean tincidunt sodales massa, et hendrerit tellus mattis ac. Sed non pretium nibh. Donec cursus maximus luctus. Vivamus lobortis eros et massa porta porttitor.

%------------------------------------------------

\subsection{Identify the regular expression in Listing \ref{lst:luftballons} and explain how it relates to the anti-war sentiments found in the rest of the script.}

Fusce varius orci ac magna dapibus porttitor. In tempor leo a neque bibendum sollicitudin. Nulla pretium fermentum nisi, eget sodales magna facilisis eu. Praesent aliquet nulla ut bibendum lacinia. Donec vel mauris vulputate, commodo ligula ut, egestas orci. Suspendisse commodo odio sed hendrerit lobortis. Donec finibus eros erat, vel ornare enim mattis et.

%----------------------------------------------------------------------------------------

\end{document}
